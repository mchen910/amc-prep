\documentclass{article}%
\usepackage[T1]{fontenc}%
\usepackage[utf8]{inputenc}%
\usepackage{lmodern}%
\usepackage{textcomp}%
\usepackage{lastpage}%
\usepackage{amsmath}%
\usepackage{amssymb}%
\usepackage{hyperref}%
\usepackage{geometry}%
\usepackage{asymptote}%
%
\title{Solutions}%
\author{MAA}%
\geometry{margin=1in}%
\date{\today}%
%
\begin{document}%
\normalsize%
\maketitle%
\section*{Acknowledgement}%
\label{sec:Acknowledgement}%
All the following problems are copyrighted by the \href{https://www.maa.org/}{Mathematical Association of America}'s \href{https://www.maa.org/math-competitions}{American Mathematics Competitions}.

%
\clearpage%
\section*{Problem 1}%
\label{sec:Problem1}%
\begin{enumerate}%
\item%
$20-2010+201+2010-201+20=20+20=\boxed{\textbf{(C)}\,40}$.

%
\end{enumerate}

%
\section*{Problem 2}%
\label{sec:Problem2}%
\begin{enumerate}%
\item%
Since T-shirts cost $5$ dollars more than a pair of socks, T-shirts cost $5+4=9$ dollars. 

Since each member needs $2$ pairs of socks and $2$ T-shirts, the total cost for $1$ member is $2(4+9)=26$ dollars. 

Since $2366$ dollars was the cost for the club, and $26$ was the cost per member, the number of members in the League is $2366\div 26=\boxed{\mathrm{(B)}\ 91}$. 

%
\end{enumerate}

%
\section*{Problem 3}%
\label{sec:Problem3}%
\begin{enumerate}%
\item%
If $a=0$ or $c=0$, the expression evaluates to $-d<0$. 
If $b=0$, the expression evaluates to $c-d\leq 2$. 
Case $d=0$ remains.
In that case, we want to maximize $c\cdot a^b$ where $\{a,b,c\}=\{1,2,3\}$. Trying out the six possibilities we get that the greatest is $(a,b,c)=(3,2,1)$, where $c\cdot a^b=1\cdot 3^2=\boxed{\mathrm{(D)}\ 9}$.

%
\end{enumerate}

%
\section*{Problem 4}%
\label{sec:Problem4}%
\begin{enumerate}%
\item%
Let $x$ be the number of points scored by the Cougars, and $y$ be the number of points scored by the Panthers. The problem is asking for the value of $y$. 
\begin{align*} x+y &= 34 \\ x-y &= 14 \\ 2x &= 48 \\ x &= 24 \\ y &= \boxed{\textbf{(A) }10} \\ \end{align*}

%
\end{enumerate}

%
\section*{Problem 5}%
\label{sec:Problem5}%
\begin{enumerate}%
\item%
After you draw $4$ socks, you can have one of each color, so (according to the \href{/wiki/index.php/Pigeonhole_principle}{pigeonhole principle}), if you pull $\boxed{\textbf{(C)}\ 5}$ then you will be guaranteed a matching pair.

%
\item%
\href{https://youtu.be/uAc9VHtRRPg?t=130}{https://youtu.be/uAc9VHtRRPg?t=130}

~IceMatrix

%
\end{enumerate}

%
\section*{Problem 6}%
\label{sec:Problem6}%
\begin{enumerate}%
\item%
It can be seen that the probability of rolling the smallest number possible is the same as the probability of rolling the largest number possible, the probability of rolling the second smallest number possible is the same as the probability of rolling the second largest number possible, and so on. This is because the number of ways to add a certain number of ones to an assortment of $7$ ones is the same as the number of ways to take away a certain number of ones from an assortment of $7$ $6$s.

So, we can match up the values to find the sum with the same probability as $10$. We can start by noticing that $7$ is the smallest possible roll and $42$ is the largest possible roll. The pairs with the same probability are as follows:

$(7, 42), (8, 41), (9, 40), (10, 39), (11, 38)...$

However, we need to find the number that matches up with $10$. So, we can stop at $(10, 39)$ and deduce that the sum with equal probability as $10$ is $39$. So, the correct answer is $\boxed{\textbf{(D)} \text{39}}$, and we are done.

Written By: Archimedes15



Add-on by ike.chen: to see how the number of ways to roll $10$ and $39$ are the same, consider this argument:

Each of the $7$ dice needs to have a nonnegative value; it follows that the number of ways to roll $10$ is $\binom {10-1}{7-1}=84$ by stars and bars. $10-7=3$, so there's no chance that any dice has a value $>6$.

Now imagine $7$ piles with $6$ blocks each. The number of ways to take $3$ blocks away (making the sum $7\cdot 6-3=39$) is also $\binom {3+7-1}{7-1}=84$.

%
\item%
Let's call the unknown value $x$. By symmetry, we realize that the difference between 10 and the minimum value of the rolls is equal to the difference between the maximum and $x$. So, 

$10 - 7 = 42- x$

$x = 39$ and our answer is  $\boxed{\textbf{(D)} \text{ 39}}$
By: Soccer\_JAMS

%
\item%
For the sums to have equal probability, the average sum of both sets of $7$ dies has to be $(6+1)\cdot 7 = 49$. Since having $10$ is similar to not having $10$, you just subtract 10 from the expected total sum. $49 - 10 = 39$ so the answer is $\boxed{\textbf{(D)} \text{ 39}}$

By: epicmonster

Revised solution by Williamgolly (includes bijections):
Notice that we first must have at least a 1 on each die.  Now, we form the following bijection: biject each value on the original die, say $x$ to a value $7-x$ on the new die. Notice how now we need to 'take away' 3 from seven dies that all show 6.  Therefore, the answer is 39.

%
\item%
The expected value of the sums of the die rolls is $3.5\cdot7=24.5$, and since the probabilities should be distributed symmetrically on both sides of $24.5$, the answer is $24.5+(24.5-10)=39$, which is $\boxed{\textbf{(D)} \text{ 39}}$.

By: dajeff

%
\item%
Another faster and easier way of doing this, without using almost any math at all, is realizing that the possible sums are ${7,8,9,10,...,39,40,41,42}$. By symmetry, (and doing a few similar problems in the past), you can realize that the probability of obtaining $7$ is the same as the probability of obtaining $42$, $P(8)=P(41)$ and on and on and on. This means that $P(10)=P(39)$, and thus the correct answer is $\boxed{\textbf{(D)} \text{ 39}}$.

By: fhdsaukfaioifk

Calculating the probability of getting a sum of $10$ is also easy. There are $3$ cases:


Case $1$: $\{1,1,1,1,1,1,4\}$


${7 \choose 6}=7$ cases


Case $2$: $\{1,1,1,1,1,2,3\}$


${7 \choose 5}=6 \cdot 7=42$ cases


Case $3$: $\{1,1,1,1,2,2,2\}$


${7 \choose 4}=\frac {7 \cdot 6 \cdot 5}{3 \cdot 2}=35$ cases


The probability is ${84 \over 6^7} = \frac{14}{6^6}$. 

Calculating $6^6$: 

$6^6=(6^3)^2=216^2=46656$

Therefore, the probability is ${14 \over 46656} = \boxed{{7 \over 23328}}$

~Zeric Hang (Main writer) and fhdsaukfaioifk (Editor)

%
\item%
There is a similar to problem 11 of the AMC 10A in the same year, which is almost an replica of the problem mentioned by Zeric Hang in the Note section:
\href{https://artofproblemsolving.com/wiki/index.php/2018_AMC_10A_Problems/Problem_11}{https://artofproblemsolving.com/wiki/index.php/2018\_AMC\_10A\_Problems/Problem\_11}

%
\item%
\href{https://youtu.be/odHniGWWLvw}{https://youtu.be/odHniGWWLvw}

~savannahsolver

%
\end{enumerate}

%
\section*{Problem 7}%
\label{sec:Problem7}%
\begin{enumerate}%
\item%
Dividing both sides by $a$, we get $x^2 - 2x + b/a = 0$. By Vieta's formulas, the sum of the roots is $2$, therefore their average is $1\Rightarrow \boxed{A}$.

%
\item%
We know that for an equation $ax^2 + bx + c = 0$, the sum of the roots is $\frac{-b}{a}$. This means that the sum of the roots for $ax^2 - 2ax + b = 0$ is $\frac{2a}{a}=2$. The average is the sum of the two roots divided by two, so the average is $\frac22 = 1 \Rightarrow \boxed{A}$.

%
\end{enumerate}

%
\section*{Problem 8}%
\label{sec:Problem8}%
\begin{enumerate}%
\item%
Since $(x-a)(x-b) = x^2 - (a+b)x + ab = x^2 + ax + b = 0$, it follows by comparing \href{/wiki/index.php/Coefficient}{coefficients} that $-a - b = a$ and that $ab = b$. Since $b$ is nonzero, $a = 1$, and $-1 - b = 1 \Longrightarrow b = -2$. Thus $(a,b) = \boxed{\mathrm{(C)}\ (1,-2)}$.

Another method is to use \href{/wiki/index.php/Vieta%27s_formulas}{Vieta's formulas}. The sum of the solutions to this polynomial is equal to the opposite of the $x$ coefficient, since the leading coefficient is 1; in other words, $a + b = -a$ and the product of the solutions is equal to the constant term (i.e, $a*b = b$). Since $b$ is nonzero, it follows that $a = 1$ and therefore (from the first equation), $b = -2a = -2$. Hence, $(a,b) = \boxed{\mathrm{(C)}\ (1,-2)}$

Since $a$ and $b$ are solutions to the given equation, we can write the two equations $a^2 + a^2 + b = 2a^2 + b = 0,$ and $b^2 + ab + b = 0.$ From the first equation, we get that $b = -2a^2,$ and substituting this in our second equation, we get that $4a^4 - 2a^3 - 2a^2 = 0,$ and solving this gives us the solutions $a = -\frac{1}{2},$ $a = 0$ and $a = 1.$ We discard the first two solutions, as the first one doesnt show up in the answer choices and we are given that $a$ is nonzero. The only valid solution is then $a = 1,$ which gives us $b = -2,$ and $(a, b) = \boxed{\text{(C) } (1, -2)}$

Note that for roots $a$ and $b$, $ab = b$. This implies that $a$ is $1$, and there is only one answer choice with $1$ in the position for $a$, hence, $(a,b) = \boxed{\mathrm{(C)}\ (1,-2)}$

%
\item%
\href{https://youtu.be/5QdPQ3__a7I?t=498}{https://youtu.be/5QdPQ3\_\_a7I?t=498}

~ pi\_is\_3.14

%
\end{enumerate}

%
\section*{Problem 9}%
\label{sec:Problem9}%
\begin{enumerate}%
\item%
We can see that $44^2=1936$ which is less than 2020. Therefore, there are $2020-44=1976$ of the $4040$ numbers greater than $2020$. Also, there are $2020+44=2064$ numbers that are less than or equal to $2020$. 

Since there are $44$ duplicates/extras, it will shift up our median's placement down $44$. Had the list of numbers been $1,2,3, \dots, 4040$, the median of the whole set would be $\dfrac{1+4040}{2}=2020.5$. 

Thus, our answer is $2020.5-44=\boxed{\textbf{(C)}\ 1976.5}$.

~aryam

~Additions by BakedPotato66

%
\item%
As we are trying to find the median of a $4040$-term set, we must find the average of the $2020$th and $2021$st terms.

Since $45^2 = 2025$ is slightly greater than $2020$, we know that the $44$ perfect squares $1^2$ through $44^2$ are less than $2020$, and the rest are greater. Thus, from the number $1$ to the number $2020$, there are $2020 + 44 = 2064$ terms. Since $44^2$ is $44 + 45 = 89$ less than $45^2 = 2025$ and $84$ less than $2020$, we will only need to consider the perfect square terms going down from the $2064$th term, $2020$, after going down $84$ terms. Since the $2020$th and $2021$st terms are only $44$ and $43$ terms away from the $2064$th term, we can simply subtract $44$ from $2020$ and $43$ from $2020$ to get the two terms, which are $1976$ and $1977$. Averaging the two, we get $\boxed{\textbf{(C)}\ 1976.5}.$

~\href{/wiki/index.php/User:Emerald_block}{emerald\_block}

%
\item%
We want to know the $2020$th term and the $2021$st term to get the median.

We know that $44^2=1936$. So, numbers $1^2, 2^2, \ldots,44^2$ are in between $1$ and $1936$.

So, the sum of $44$ and $1936$ will result in $1980$, which means that $1936$ is the $1980$th number.

Also, notice that $45^2=2025$, which is larger than $2021$.

Then the $2020$th term will be $1936+40 = 1976$, and similarly the $2021$th term will be $1977$.

Solving for the median of the two numbers, we get $\boxed{\textbf{(C)}\ 1976.5}$

~toastybaker

%
\item%
We note that $44^2 = 1936$, which is the first square less than $2020$, which means that there are $44$ additional terms before $2020$. This makes $2020$ the $2064$th term. To find the median, we need the $2020$th and $2021$st term. We note that every term before $2020$ is one less than the previous term (That is, we subtract $1$ to get the previous term.). If $2020$ is the $2064$th term, than $2020 - 44$ is the $(2064 - 44)$th term. So, the $2020$th term is $1976$, and the $2021$st term is $1977$, and the average of these two terms is the median, or $\boxed{\textbf{(C)}\ 1976.5}$.

~primegn

%
\item%
To find the median, we sort the $4040$ numbers in decreasing order, then average the $2020$th and the $2021$st numbers of the sorted list.

Since $45^2=2025$ and $44^2=1936,$ the first $2021$ numbers of the sorted list are \[\underbrace{2020^2,2019^2,2018^2,\ldots,46^2,45^2}_{1976\mathrm{ \ numbers}}\phantom{ },\phantom{ }\underbrace{2020,2019,2018,\ldots,1977,1976}_{45\mathrm{ \ numbers}}\phantom{ },\] from which the answer is $\frac{1977+1976}{2}=\boxed{\textbf{(C)}\ 1976.5}.$

~MRENTHUSIASM

%
\item%
\href{https://youtu.be/luMQHhp_Rfk}{https://youtu.be/luMQHhp\_Rfk}

Education, The Study of Everything

%
\item%
\href{https://youtu.be/ZGwAasE32Y4}{https://youtu.be/ZGwAasE32Y4}

~IceMatrix

%
\item%
\href{https://youtu.be/B0RPkcjdkPU}{https://youtu.be/B0RPkcjdkPU}

~savannahsolver

%
\end{enumerate}

%
\section*{Problem 10}%
\label{sec:Problem10}%
\begin{enumerate}%
\item%
\[\begin{split}& (n+1)n! + (n+2)(n+1)n! = 440 \cdot n! \\ \Rightarrow \ &n![n+1 + (n+2)(n+1)] = 440 \cdot n! \\ \Rightarrow \ &n + 1 + n^2 + 3n + 2 = 440 \\ \Rightarrow \ &n^2 + 4n - 437 = 0\end{split}\]

Solving by the quadratic formula, $n = \frac{-4\pm \sqrt{16+437\cdot4}}{2} = \frac{-4\pm 42}{2} = \frac{38}{2} = 19$ (since clearly $n \geq 0$). The answer is therefore $1 + 9 = \boxed{\textbf{(C) }10}$.

Dividing both sides by $n!$ gives
\[(n+1)+(n+2)(n+1)=440 \Rightarrow n^2+4n-437=0 \Rightarrow (n-19)(n+23)=0.\]
Since $n$ is non-negative, $n=19$. The answer is $1 + 9 = \boxed{\textbf{(C) }10}$.

Dividing both sides by $n!$ as before gives $(n+1)+(n+1)(n+2)=440$. Now factor out $(n+1)$, giving $(n+1)(n+3)=440$. By considering the prime factorization of $440$, a bit of experimentation gives us $n+1=20$ and $n+3=22$, so $n=19$, so the answer is $1 + 9 = \boxed{\textbf{(C) }10}$.

Since $(n+1)! + (n+2)! = (n+1)n! + (n+2)(n+1)n! = 440 \cdot n!$, the result can be factored into $(n+1)(n+3)n!=440 \cdot n!$ and divided by $n!$ on both sides to get $(n+1)(n+3)=440$. From there, it is easier to complete the square with the quadratic $(n+1)(n+3) = n^2 + 4n + 3$, so $n^2+4n+4=441 \Rightarrow (n+2)^2=441$. Solving for $n$ results in $n=19,-23$, and since $n>0$, $n=19$ and the answer is $1 + 9 = \boxed{\textbf{(C) }10}$.

~Randomlygenerated

%
\item%
\href{https://youtu.be/ba6w1OhXqOQ?t=1956}{https://youtu.be/ba6w1OhXqOQ?t=1956}

~ pi\_is\_3.14

%
\item%
\href{https://youtu.be/7xf_g3YQk00}{https://youtu.be/7xf\_g3YQk00}

~IceMatrix

\href{https://youtu.be/6YFN_hwotUk}{https://youtu.be/6YFN\_hwotUk}

~savannahsolver

%
\end{enumerate}

%
\section*{Problem 11}%
\label{sec:Problem11}%
\begin{enumerate}%
\item%
Let $x$ be the probability of flipping heads. It follows that the probability of flipping tails is $1-x$.

The probability of flipping $2$ heads and $2$ tails is equal to the number of ways to flip it times the product of the probability of flipping each coin.

\begin{align*}{4 \choose 2}x^2(1-x)^2 &= \frac{1}{6}\\ 6x^2(1-x)^2 &= \frac{1}{6}\\ x^2(1-x)^2 &= \frac{1}{36}\\ x(1-x) &= \pm\frac{1}{6}\end{align*}

As for the desired probability $x$ both $x$ and $1-x$ are nonnegative, we only need to consider the positive root, hence

\begin{align*}x(1-x) &= \frac{1}{6}\\ 6x^2-6x+1&=0\end{align*}

Applying the quadratic formula we get that the roots of this equation are $\frac{3\pm\sqrt{3}}{6}$. As the probability of heads is less than $\frac{1}{2}$, we get that the answer is $\boxed{\textbf{(D)}\ \frac{3-\sqrt{3}}{6}}$.

%
\end{enumerate}

%
\section*{Problem 12}%
\label{sec:Problem12}%
\begin{enumerate}%
\item%
Let total distance be $x$. Her speed in miles per minute is $\frac{x}{180}$. Then, the distance that she drove before hitting the snowstorm is $\frac{x}{3}$. Her speed in snowstorm is reduced $20$ miles per hour, or $\frac{1}{3}$ miles per minute. Knowing it took her $276$ minutes in total, we create equation: 
\[\text{Time before Storm}\, + \, \text{Time after Storm} = \text{Total Time} \Longrightarrow\]
\[\frac{\text{Distance before Storm}}{\text{Speed before Storm}} + \frac{\text{Distance in Storm}}{\text{Speed in Storm}} = \text{Total Time} \Longrightarrow \frac{\frac{x}{3}}{\frac{x}{180}} + \frac{\frac{2x}{3}}{\frac{x}{180} - \frac{1}{3}} = 276\]

Solving equation, we get $x=135$ $\Longrightarrow \boxed{B}$.

%
\item%
\href{https://youtu.be/N4MC_a4Z_2k}{https://youtu.be/N4MC\_a4Z\_2k}

~savannahsolver

%
\end{enumerate}

%
\section*{Problem 13}%
\label{sec:Problem13}%
\begin{enumerate}%
\item%
The equation of the circle is $(x-5)^2+(y-5)^2=100$. Plugging in the given conditions we have $(x-5)^2+(-x-5)^2 \leq 100$. Expanding gives: $x^2-10x+25+x^2+10x+25\leq 100$, which simplifies to
$x^2\leq 25$ and therefore
$x\leq 5$ and $x\geq -5$. So $x$ ranges from $-5$ to $5$, for a total of $\boxed{\mathbf{(A)}\ 11}$ integer values.

Note by Williamgolly:
Alternatively, draw out the circle and see that these points must be on the line $y=-x$.

%
\end{enumerate}

%
\section*{Problem 14}%
\label{sec:Problem14}%
\begin{enumerate}%
\item%
We have $b-a=c-b$, so $a=2b-c$. Since $a,c,b$ is geometric, $c^2=ab=(2b-c)b \Rightarrow 2b^2-bc-c^2=(2b+c)(b-c)=0$. Since $a<b<c$, we can't have $b=c$ and thus $c=-2b$. Then our arithmetic progression is $4b,b,-2b$. Since $4b < b < -2b$, $b < 0$. The smallest possible value of $c=-2b$ is $(-2)(-1)=2$, or $\boxed{\textbf{(C)}}$.

(Solution by AT)

%
\item%
Taking the definition of an arithmetic progression, there must be a common difference between the terms, giving us $(b-a) = (c-b)$. From this, we can obtain the expression $a = 2b-c$. Again, by taking the definition of a geometric progression, we can obtain the expression, $c=ar$ and $b=ar^2$, where r serves as a value for the ratio between two terms in the progression. By substituting $b$ and $c$ in the arithmetic progression expression with the obtained values from the geometric progression, we obtain the equation, $a=2ar^2-ar$ which can be simplified to $(r-1)(2r+1)=0$ giving us $r=1$ or $r=-1/2$. Thus, from the geometric progression, $a=a$, $b=1/4a$ and $c=-1/2a$. Looking at the initial conditions of $a<b<c$ we can see that the lowest integer value that would satisfy the above expressions is if $a = -4$, thus making $c=2$ or $\boxed{\textbf{(C)}}$ 
(Solution by thatuser)

%
\item%
By the definition of an arithmetic progression, we can label the terms $a$, $b$, and $c$, as $a$, $a+d$, and $a+2d$. Now, we have that $a$, $a+2d$, and $a+d$ form a geometric progression. Since a geometric ratio has a common ratio between terms, we have $(a+2d)/a = (a+d)/a+2d$. Cross multiplying, we obtain the equation $(a+2d)^2=a(a+d)$. Multiplying it out and cancelling terms, we are left with the quadratic equation $4d^2+3ad=0$. Solving for $d$ in terms of $a$, we get that $d=-3a/4$ or $d=0$. Looking at the problem, we know that the $d$ cannot be 0, therefore the arithmetic progression is $a, a/4, -a/2$, so we need to find the minimum value of $-a/2$ while $-a/2>a$. Looking at our progression, we realize that a must be a multiple of 4 so that every term is an integer. Substituting $a=-4$, since that would yield the smallest value of $-a/2$ which satisfies the conditions, we figure out that the answer is $\boxed{\textbf{(C)}}$.
(Solution by i8Pie)

%
\end{enumerate}

%
\section*{Problem 15}%
\label{sec:Problem15}%
\begin{enumerate}%
\item%
The domain over which we solve the equation is $\mathbb{R} \setminus \{2,6\}$.

We can now cross-multiply to get rid of the fractions, we get $(x-1)(x-6)=(x-k)(x-2)$.

Simplifying that, we get $7x-6 = (k+2)x - 2k$. Clearly for $k=5$ we get the equation $-6=-10$ which is never true. The answer is $\boxed{\mathrm{ (E)}\ 5}$

For other $k$, one can solve for $x$: $x(5-k) = 6-2k$, hence $x=\frac {6-2k}{5-k}$. We can easily verify that for none of the other 4 possible values of $k$ is this equal to $2$ or $6$, hence there is a solution for $x$ in each of the other cases.

-Edited by XxHalo711 (typo within the solution)

%
\end{enumerate}

%
\section*{Problem 16}%
\label{sec:Problem16}%
\begin{enumerate}%
\item%
By symmetry, the probability of the red ball landing in a higher-numbered bin is the same as the probability of the green ball landing in a higher-numbered bin. Clearly, the probability of both landing in the same bin is $\sum_{k=1}^{\infty}{2^{-k} \cdot 2^{-k}} = \sum_{k=1}^{\infty}2^{-2k} = \frac{1}{3}$ (by the geometric series sum formula). Therefore, since the other two probabilities have to both the same, they have to be $\frac{1-\frac{1}{3}}{2} = \boxed{\textbf{(C) } \frac{1}{3}}$.

Suppose the green ball goes in bin $i$, for some $i \ge 1$. The probability of this occurring is $\frac{1}{2^i}$. Given that this occurs, the probability that the red ball goes in a higher-numbered bin is $\frac{1}{2^{i+1}} + \frac{1}{2^{i+2}} + \ldots = \frac{1}{2^i}$ (by the geometric series sum formula). Thus the probability that the green ball goes in bin $i$, and the red ball goes in a bin greater than $i$, is $\left(\frac{1}{2^i}\right)^2 = \frac{1}{4^i}$. Summing from $i=1$ to infinity, we get

\[\sum_{i=1}^{\infty} \frac{1}{4^i} = \boxed{\textbf{(C) } \frac{1}{3}}\]
where we again used the geometric series sum formula. (Alternatively, if this sum equals $n$, then by writing out the terms and multiplying both sides by $4$, we see $4n = n+1$, which gives $n = \frac{1}{3}$.)

For red ball in bin $k$, $\Pr(\text{Green Below Red})=\sum\limits_{i=1}^{k-1}2^{-i}$ (GBR) and $\Pr(\text{Red in Bin k}=2^{-k}$ (RB). 
\[\Pr(\text{GBR}|\text{RB})=\sum\limits_{k=1}^{\infty}2^{-k}\sum\limits_{i=1}^{k-1}2^{-i}=\sum\limits_{k=1}^{\infty}2^{-k}\cdot\frac{1}{2}(\frac{1-(1/2)^{k-1}}{1-1/2})\]
\[\sum\limits_{k=1}^{\infty}\frac{1}{2^{-k}}-2\sum\limits_{k=1}^\infty\frac{1}{(2^2)^{-k}}\implies 1-2/3=\boxed{(\textbf{C}) \frac{1}{3}}\]

The probability that the two balls will go into adjacent bins is $\frac{1}{2\times4} + \frac{1}{4\times8} + \frac{1}{8 \times 16} + ... = \frac{1}{8} + \frac{1}{32} + \frac{1}{128} + \cdots = \frac{1}{6}$ by the geometric series sum formula. Similarly, the probability that the two balls will go into bins that have a distance of $2$ from each other is $\frac{1}{2 \times 8} + \frac{1}{4 \times 16} + \frac{1}{8 \times 32} + \cdots = \frac{1}{16} + \frac{1}{64} + \frac{1}{256} + \cdots = \frac{1}{12}$ (again recognizing a geometric series). We can see that each time we add a bin between the two balls, the probability halves. Thus, our answer is $\frac{1}{6} + \frac{1}{12} + \frac{1}{24} + \cdots$, which, by the geometric series sum formula, is $\boxed{\textbf{(C) } \frac{1}{3}}$.
-fidgetboss\_4000

Define a win as a ball appearing in higher numbered box.

Start from the first box. 

There are $4$ possible results in the box: Red, Green, Red and Green, or none, with an equal probability of $\frac{1}{4}$ for each. If none of the balls is in the first box, the game restarts at the second box with the same kind of probability distribution, so if $p$ is the probability that Red wins, we can write $p = \frac{1}{4} + \frac{1}{4}p$: there is a $\frac{1}{4}$ probability that "Red" wins immediately, a $0$ probability in the cases "Green" or "Red and Green", and in the "None" case (which occurs with $\frac{1}{4}$ probability), we then start again, giving the same probability $p$. Hence, solving the equation, we get $p = \boxed{\textbf{(C) } \frac{1}{3}}$.

Write out the infinite geometric series as $\frac{1}{2}$, $\frac{1}{4}, \frac{1}{8}, \frac{1}{16}, \cdots$. To find the probablilty that red goes in a higher-numbered bin than green, we can simply remove all odd-index terms (i.e term $1$, term $3$, etc.), and then sum the remaining terms - this is in fact precisely equivalent to the method of Solution 2. Writing this out as another infinite geometric sequence, we are left with $\frac{1}{4}, \frac{1}{16}, \frac{1}{64}, \cdots$. Summing, we get \[\sum_{i=1}^{\infty} \frac{1}{4^i} = \boxed{\textbf{(C) } \frac{1}{3}}\]

Fixing the green ball to fall into bin $1$ gives a probability of $\frac{1}{2}\left(\frac{1}{2^2}+\frac{1}{2^3} +...\right)$ for the red ball to fall into a higher bin. Fixing the green ball to fall into bin $2$ gives a probability of $\frac{1}{2^2}\left(\frac{1}{2^3}+\frac{1}{2^4} +...\right)$. Factoring out the denominator of the first fraction in each probability gives $\frac{1}{2^3}\left(1+\frac{1}{2}+\frac{1}{2^2}+...\right)+\frac{1}{2^5}\left(1+\frac{1}{2}+\frac{1}{2^2}+...\right)+...$ so factoring out $\left(1+\frac{1}{2}+\frac{1}{2^2}+\frac{1}{2^3}+...\right)$ results in the probability simplifying to $\left(\frac{1}{2^3}+\frac{1}{2^5}+\frac{1}{2^7}+...\right)\left(1+\frac{1}{2}+\frac{1}{2^2}+\frac{1}{2^3}+...\right)$ and using the formula $\frac{a}{1-r}$ to find both series, we obtain $\left(\frac{\frac{1}{2^3}}{1-\frac{1}{4}}\right)\left(\frac{1}{1-\frac{1}{2}}\right)$ which simplifies to $\boxed{\textbf{(C) } \frac{1}{3}}$ -- OGBooger

We can think of this problem as "what is the probability that the green ball's bin is less than the red ball's bin". We do not consider the case where the red ball goes into bin $1$ because the green ball has no where to go then. The chance that the green one is below the red one if the red one goes to bin $2$ is $\frac{1}{4}$ chance that the red ball even goes in bin $2$ and $\frac{1}{2}$ chance that the green ball goes into any bin less than $2$. Similarly, if the red goes into bin $3$, there is a $\frac{1}{8} \cdot \left(\frac{1}{4} + \frac{1}{2}\right)$ chance, or $\frac{3}{32}$, continuing like this, we get this sequence:

$\frac{1}{8}, \frac{3}{32}, \frac{7}{128}, ...$

Let $S$ equal the sum of our series:

$S = \frac{1}{8} + \frac{3}{32} + \frac{7}{128} + ...$. That means we can write another equation:
$\frac{S}{4} = \frac{1}{32} + \frac{3}{128} + ...$

Subtracting $\frac{S}{4}$ from $S$, yields:

$S - \frac{S}{4} = \frac{1}{8} + \frac{2}{32} + \frac{4}{128} + ...$

We see that the above series is a infinite geometric sequence with common ratio $\frac{1}{2}$. Therefore, the sum of that infinite series is $\frac{\frac{1}{8}}{\frac{1}{2}}$, which equals $\frac{1}{4}$. Our equation is now $S - \frac{S}{4} = \frac{1}{4}$. Solving for $S$ shows that $S = \frac{1}{3}$.

Our answer is $\boxed{\textbf{(C) }\frac{1}{3}}$

~ericshi1685

Denote $G,R$ the bin numbers of the green and red balls, respectively. The common probability distribution of $G,B$ can be constructed by keep splitting the remaining unassigned probability into two halves: one goes to the smallest number that has not been assigned, and the other goes to the rest. In other words, $\Pr(G=k) = \Pr (G>k), \forall k \in \mathbb{N}$. Then,

\[\Pr(G>R)=\sum_{k=1}^\infty \Pr(G>k) \Pr(R=k) = \sum_{k=1}^\infty \Pr(G=k) \Pr(R=k) = \Pr (G=R)\]

Similarly $\Pr(G<R)=\Pr(G=R)$. Therefore all three probabilities equal $\boxed{\textbf{(C) }\frac{1}{3}}$.

~asops

%
\item%
For those who want a video solution: \href{https://youtu.be/VP7ltu-XEq8}{https://youtu.be/VP7ltu-XEq8}

\href{https://youtu.be/_0YaCyxiMBo?t=353}{https://youtu.be/\_0YaCyxiMBo?t=353}

~IceMatrix

\href{https://youtu.be/IRyWOZQMTV8?t=2484}{https://youtu.be/IRyWOZQMTV8?t=2484}

~ pi\_is\_3.14

%
\end{enumerate}

%
\section*{Problem 17}%
\label{sec:Problem17}%
\begin{enumerate}%
\item%

\begin{center}
\begin{asy}
	import olympiad; import cse5;   size(8cm,8cm); path circ1, circ2; circ1=circle((0,0),5); circ2=circle((3,4),3); pair O, Z; O=(3,4); Z=(3,-4); pair [] x=intersectionpoints(circ1,circ2); pair [] y=intersectionpoints(x[1]--Z,circ2); pair B; B=midpoint(x[1]--y[0]); draw(B--O); draw(x[0]--Z); draw(O--Z); draw(x[1]--Z); draw(O--x[0]); draw(circ1); draw(circ2); draw(rightanglemark(Z,B,O,15)); draw(x[1]--O--y[0]); label("$O$",O,NE); label("$Y$",x[0],SE); label("$X$",x[1],NW); label("$Z$",Z,S); label("$A$",y[0],SW); label("$B$",B,SW);
\end{asy}
\end{center}

%
\item%
Let $r$ denote the radius of circle $C_1$. Note that quadrilateral $ZYOX$ is cyclic. By Ptolemy's Theorem, we have $11XY=13r+7r$ and $XY=20r/11$. Let $t$ be the measure of angle $YOX$. Since $YO=OX=r$, the law of cosines on triangle $YOX$ gives us $\cos t =-79/121$. Again since $ZYOX$ is cyclic, the measure of angle $YZX=180-t$. We apply the law of cosines to triangle $ZYX$ so that $XY^2=7^2+13^2-2(7)(13)\cos(180-t)$. Since $\cos(180-t)=-\cos t=79/121$ we obtain $XY^2=12000/121$. But$XY^2=400r^2/121$ so that $r=\boxed{(E)\sqrt{30}}$.

%
\item%
Let us call the $r$ the radius of circle $C_1$, and $R$ the radius of $C_2$. Consider $\triangle OZX$ and $\triangle OZY$. Both of these triangles have the same circumcircle ($C_2$). From the Extended Law of Sines, we see that $\frac{r}{\sin{\angle{OZY}}} = \frac{r}{\sin{\angle{OZX}}}= 2R$. Therefore, $\angle{OZY} \cong \angle{OZX}$. We will now apply the Law of Cosines to $\triangle OZX$ and $\triangle OZY$ and get the equations 

$r^2 = 13^2 + 11^2 - 2 \cdot 13 \cdot 11 \cdot \cos{\angle{OZX}}$,

$r^2 = 11^2 + 7^2 - 2 \cdot 11 \cdot 7 \cdot \cos{\angle{OZY}}$, 

respectively. Because $\angle{OZY} \cong \angle{OZX}$, this is a system of two equations and two variables. Solving for $r$ gives $r = \sqrt{30}$. $\boxed{E}$.

Instead of using the Extended Law of Sines, you can note that $OX = OY \implies \text{arc}\ OX =\text{arc}\ OY \implies \angle{OZY} \cong \angle{OZX}$, since the angles inscribe arcs of the same length.

%
\item%
Let $r$ denote the radius of circle $C_1$. Note that quadrilateral $ZYOX$ is cyclic. By Ptolemy's Theorem, we have $11XY=13r+7r$ and $XY=20r/11$. Consider isosceles triangle $XOY$. Pulling an altitude to $XY$ from $O$, we obtain $\cos(\angle{OXY}) = \frac{10}{11}$. Since quadrilateral $ZYOX$ is cyclic, we have $\angle{OXY}=\angle{OZY}$, so $\cos(\angle{OXY}) = \cos(\angle{OZY})$. Applying the Law of Cosines to triangle $OZY$, we obtain $\frac{10}{11} = \frac{7^2+11^2-r^2}{2(7)(11)}$. Solving gives $r=\sqrt{30}$. $\boxed{E}$.

-Solution by thecmd999

%
\item%
Let $P = XY \cap OZ$. Consider an inversion about $C_1 \implies C_2 \to XY, Z \to P$. So,  $OP \cdot OZ = r^2 \implies OP = r^2/11 \implies PZ = \dfrac{121 - r^2}{11}$. Using $\triangle YPZ \sim OXZ \implies  r = \sqrt{30} \implies \boxed{E}$.


-Solution by IDMasterz

%
\item%
Notice that $\angle YZO=\angle XZO$ as they subtend arcs of the same length. Let $A$ be the point of intersection of $C_1$ and $XZ$. We now have $AZ=YZ=7$ and $XA=6$. Furthermore, notice that $\triangle XAO$ is isosceles, thus the altitude from $O$ to $XA$ bisects $XZ$ at point $B$ above. By the Pythagorean Theorem, \begin{align*}BZ^2+BO^2&=OZ^2\\(BA+AZ)^2+OA^2-BA^2&=11^2\\(3+7)^2+r^2-3^2&=121\\r^2&=30\end{align*}Thus, $r=\sqrt{30}\implies\boxed{\textbf{E}}$

%
\item%
Use the diagram above. Notice that $\angle YZO=\angle XZO$ as they subtend arcs of the same length. Let $A$ be the point of intersection of $C_1$ and $XZ$. We now have $AZ=YZ=7$ and $XA=6$. Consider the power of point $Z$ with respect to Circle $O,$ we have $13\cdot 7 = (11 + r)(11 - r) = 11^2 - r^2,$ which gives $r=\boxed{\sqrt{30}}.$

%
\item%
Note that $OX$ and $OY$ are the same length, which is also the radius $R$ we want. Using the law of cosines on $\triangle OYZ$, we have $11^2=R^2+7^2-2\cdot 7 \cdot R \cdot \cos\theta$, where $\theta$ is the angle formed by $\angle{OYZ}$. Since $\angle{OYZ}$ and $\angle{OXZ}$ are supplementary, $\angle{OXZ}=\pi-\theta$. Using the law of cosines on $\triangle OXZ$, $11^2=13^2+R^2-2 \cdot 13 \cdot R \cdot \cos(\pi-\theta)$. As $\cos(\pi-\theta)=-\cos\theta$, $11^2=13^2+R^2+\cos\theta$. Solving for theta on the first equation and substituting gives $\frac{72-R^2}{14R}=\frac{48+R^2}{26R}$. Solving for R gives $R=\textbf{(E)}\ \boxed{\sqrt{30}}$.

%
\item%
We first note that $C_2$ is the circumcircle of both $\triangle XOZ$ and $\triangle OYZ$. Thus the circumradius of both the triangles are equal. We set the radius of $C_1$ as $r$, and noting that the circumradius of a triangle is $\frac{abc}{4A}$ and that the area of a triangle by Heron's formula is $\sqrt{(S)(S-a)(S-b)(S-c)}$ with $S$ as the semi-perimeter we have the following, \begin{align*}\dfrac{r \cdot 13 \cdot 11}{4\sqrt{(12 + \frac{r}{2})(12 - \frac{r}{2})(1 + \frac{r}{2})(\frac{r}{2} - 1)}} &= \dfrac{r \cdot 7 \cdot 11}{4\sqrt{(9 + \frac{r}{2})(9 - \frac{r}{2})(2 + \frac{r}{2})(\frac{r}{2} - 2)}} \\ \dfrac{13}{\sqrt{(144- \frac{r^2}{4})(\frac{r^2}{4} - 1)}} &= \dfrac{7}{\sqrt{(81- \frac{r^2}{4})(\frac{r^2}{4} - 4)}} \\ 169 \cdot (81 - \frac{r^2}{4})(\frac{r^2}{4} - 4) &= 49 \cdot (144 - \frac{r^2}{4})(\frac{r^2}{4} - 1) .\end{align*}
Now substituting $a = \frac{r^2}{4}$, \begin{align*}169a^2 - (85) \cdot 169 a + 4 \cdot 81 \cdot 169 &= 49a^2 - (145) \cdot 49 a + 144 \cdot 49 \\ 120a^2 - 7260a + 47700 &= 0 \\ 2a^2 - 121a + 795 &= 0 \\ (2a-15)(a-53) &= 0 \\ a = \frac{15}{2}, 53.\end{align*}
This gives us 2 values for $r$ namely $r = \sqrt{4 \cdot \frac{15}{2}} = \sqrt{30}$ and $r = \sqrt{4 \cdot 53} = 2\sqrt{53}$.

Now notice that we can apply Ptolemy's theorem on $XOYZ$ to find $XY$ in terms of $r$. We get \begin{align*}11 \cdot XY &= 13r + 7r \\ XY &= \frac{20r}{11}.\end{align*}
Here we substitute our $2$ values of $r$ receiving $XY = \frac{20\sqrt{30}}{11}, \frac{40\sqrt{53}}{11}$. Notice that the latter of the $2$ cases does not satisfy the triangle inequality for $\triangle XYZ$ as $\frac{40\sqrt{53}}{11} \approx 26.5 > 7 + 13 = 20$. But the former does thus our answer is $\textbf{(E)}\ \boxed{\sqrt{30}}$. 

~Aaryabhatta1

%
\end{enumerate}

%
\section*{Problem 18}%
\label{sec:Problem18}%
\begin{enumerate}%
\item%
Rearranging, we get $a+8c=7b+4$ and $8a-c=7-4b$

Squaring both, $a^2+16ac+64c^2=49b^2+56b+16$ and $64a^2-16ac+c^2=16b^2-56b+49$ are obtained.

Adding the two equations and dividing by $65$ gives $a^2+c^2=b^2+1$, so $a^2-b^2+c^2=\boxed{(\text{B})1}$.

The easiest way is to assume a value for $a$ and then solve the system of equations. For $a = 1$, we get the equations 
$-7b + 8c = 3$ and
$4b - c = -1$
Multiplying the second equation by $8$, we have 
$32b - 8c = -8$
Adding up the two equations yields 
$25b = -5$, so $b = -\frac{1}{5}$
We obtain $c = \frac{1}{5}$ after plugging in the value for $b$.
Therefore, $a^2-b^2+c^2 = 1-\frac{1}{25}+\frac{1}{25}=\boxed{1}$ which corresponds to $\text{(B)}$.
This time-saving trick works only because we know that for any value of $a$, $a^2-b^2+c^2$ will always be constant (it's a contest), so any value of $a$ will work.

%
\end{enumerate}

%
\section*{Problem 19}%
\label{sec:Problem19}%
\begin{enumerate}%
\item%
Inscribe circle $C$ of radius $r$ inside triangle $ABC$ so that it meets $AB$ at $Q$, $BC$ at $R$, and $AC$ at $S$. Note that angle bisectors of triangle $ABC$ are concurrent at the center $O$(also $I$) of circle $C$. Let $x=QB$, $y=RC$ and $z=AS$. Note that $BR=x$, $SC=y$ and $AQ=z$. Hence $x+z=27$, $x+y=25$, and $z+y=26$. Subtracting the last 2 equations we have $x-z=-1$ and adding this to the first equation we have $x=13$. 

By Heron's formula for the area of a triangle we have that the area of triangle $ABC$ is $\sqrt{39(14)(13)(12)}$. On the other hand the area is given by $(1/2)25r+(1/2)26r+(1/2)27r$.  Then $39r=\sqrt{39(14)(13)(12)}$ so that $r^2=56$.

Since the radius of circle $O$ is perpendicular to $BC$ at $R$, we have by the pythagorean theorem $BO^2=BI^2=r^2+x^2=56+169=225$ so that $BI=\boxed{\textbf{(A) } 15}$.

%
\item%
We can use mass points and Stewart's to solve this problem. Because we are looking at the Incenter we then label $A$ with a mass of $25$, $B$ with $26$, and $C$ with $27$. We also label where the angle bisectors intersect the opposite side $A'$, $B'$, and $C'$ correspondingly. It follows then that point $B'$ has mass $52$. Which means that $\overline{BB'}$ is split into a $2:1$ ratio. We can then use Stewart's to find $\overline{BB'}$. So we have $25^2\frac{27}{2} + 27^2\frac{25}{2} = \frac{25 \cdot 26 \cdot 27}{4} + 26\overline{BB'}^2$. Solving we get $\overline{BB'} = \frac{45}{2}$. Plugging it in we get $\overline{BI} = 15$. Therefore the answer is $\boxed{\textbf{(A) } 15}$

-Solution by arowaaron

%
\item%
We can use POP(Power of a point) to solve this problem. First, notice that the area of $\triangle ABC$ is $\sqrt{39(39 - 27)(39 - 26)(39 - 25)} = 78\sqrt{14}$. Therefore, using the formula that $sr = A$, where $s$ is the semi-perimeter and $r$ is the length of the inradius, we find that $r = 2\sqrt{14}$. 

Draw radii to the three tangents, and let the tangent hitting $BC$ be $T_1$, the tangent hitting $AB$ be $T_2$, and the tangent hitting $AC$ be $T_3$. Let $BI = x$. By the pythagorean theorem, we know that $BT_1 = \sqrt{x^2 - 56}$. By POP, we also know that $BT_2$ is also $\sqrt{x^2 - 56}$. Because we know that $BC = 25$, we find that $CT_1 = 25 - \sqrt{x^2 - 56}$. We can rinse and repeat and find that $AT_2 = 26 - (25 - \sqrt{x^2 - 56}) = 1 + \sqrt{x^2 - 56}$. We can find $AT_2$ by essentially coming in from the other way. Since $AB = 27$, we also know that $AT_3 = 27 - \sqrt{x^2 - 56}$. By POP, we know that $AT_2 = AT_3$, so $1 + \sqrt{x^2 - 56} = 27 - \sqrt{x^2 - 56}$.

Let $\sqrt{x^2 - 56} = A$, for simplicity. We can change the equation into $1 + A = 27 - A$, which we find $A$ to be $13$. Therefore, $\sqrt{x^2 - 56} = 13$, which further implies that $x^2 - 56 = 169$. After simplifying, we find $x^2 = 225$, so $x = \boxed{\textbf{(A) } 15}$

~EricShi1685

%
\end{enumerate}

%
\section*{Problem 20}%
\label{sec:Problem20}%
\begin{enumerate}%
\item%
Let $L_1$ be the line that goes through $(2,4)$ and $(14,9)$, and let $L_2$ be the line $y=mx+b$. If we let $\theta$ be the measure of the acute angle formed by $L_1$ and the x-axis, then $\tan\theta=\frac{5}{12}$. $L_1$ clearly bisects the angle formed by $L_2$ and the x-axis, so $m=\tan{2\theta}=\frac{2\tan\theta}{1-\tan^2{\theta}}=\frac{120}{119}$. We also know that $L_1$ and $L_2$ intersect at a point on the x-axis. The equation of $L_1$ is $y=\frac{5}{12}x+\frac{19}{6}$, so the coordinate of this point is $\left(-\frac{38}{5},0\right)$. Hence the equation of $L_2$ is $y=\frac{120}{119}x+\frac{912}{119}$, so $b=\frac{912}{119}$, and our answer choice is $\boxed{\mathrm{E}}$.

%
\end{enumerate}

%
\section*{Problem 21}%
\label{sec:Problem21}%
\begin{enumerate}%
\item%
We say Andy's lawn has an area of $x$. Beth's lawn thus has an area of $\frac{x}{2}$, and Carlos's lawn has an area of $\frac{x}{3}$. 

We say Andy's lawn mower cuts at a speed of $y$. Carlos's cuts at a speed of $\frac{y}{3}$, and Beth's cuts at a speed $\frac{2y}{3}$.

Each person's lawn is cut at a time of $\frac{\text{area}}{\text{rate}}$, so Andy's is cut in $\frac{x}{y}$ time, as is Carlos's. Beth's is cut in $\frac{3}{4}\times\frac{x}{y}$, so the first one to finish is $\boxed{\mathrm{(B)}\ \text{Beth}}$.



%
\item%
WLOG, we can set values of their lawns' areas and their owners' speeds. Let the area of Andy's lawn be $6$ units, Beth's lawn be $3$ units, and Carlos's lawn be $2$ units. Let Carlos's mowing area per hour (honestly the time you set won't matter) be $1$, let Beth's mowing area per hour be $2$, and let Andy's mowing area per hour be $3$. Now, we can easily calculate their time  by dividing their lawns' areas by their respective owners' speeds. Andy's time is $\frac{6}{3} = 2$ hours, Beth's time is $\frac{3}{2} = 1.5$ hours, and Carlos's time is $\frac{6}{3} = 2$ hours. Our answer is clearly $\boxed{\mathrm{(B)}\ \text{Beth}}$.


~Solution by virjoy2001

%
\end{enumerate}

%
\section*{Problem 22}%
\label{sec:Problem22}%
\begin{enumerate}%
\item%
We first want to find out which sequences of coin flips satisfy the given condition. For Debra to see the second tail before the second head, her first flip can't be heads, as that would mean she would either end with double tails before seeing the second head, or would see two heads before she sees two tails. Therefore, her first flip must be tails. The shortest sequence of flips by which she can get two heads in a row and see the second tail before she sees the second head is $THTHH$, which has a probability of $\frac{1}{2^5} = \frac{1}{32}$. Furthermore, she can prolong her coin flipping by adding an extra $TH$, which itself has a probability of $\frac{1}{2^2} = \frac{1}{4}$. Since she can do this indefinitely, this gives an infinite geometric series, which means the answer (by the geometric series sum formula) is $\frac{\frac{1}{32}}{1-\frac{1}{4}} = \boxed{\textbf{(B) }\frac{1}{24}}$.

%
\item%
Note that the sequence must start in THT, which happens with $\frac{1}{8}$ probability. Now, let $P$ be the probability that Debra will get two heads in a row after flipping THT. Either Debra flips two heads in a row immediately (probability $\frac{1}{4}$), or flips a head and then a tail and reverts back to the "original position" (probability $\frac{1}{4}P$). Therefore, $P=\frac{1}{4}+\frac{1}{4}P$, so $P=\frac{1}{3}$, so our final answer is $\frac{1}{8}\times\frac{1}{3}=\boxed{\textbf{(B) }\frac{1}{24}}$.   -Stormersyle

%
\item%
\href{https://youtu.be/wopflrvUN2c?t=993}{https://youtu.be/wopflrvUN2c?t=993}

\href{https://www.youtube.com/watch?v=2f1zEvfUe9o}{https://www.youtube.com/watch?v=2f1zEvfUe9o}

%
\end{enumerate}

%
\section*{Problem 23}%
\label{sec:Problem23}%
\begin{enumerate}%
\item%
Let $x$ be the \href{/wiki/index.php/Probability}{probability} of rolling a $1$. The probabilities of rolling a 
$2$, $3$, $4$, $5$, and $6$ are $2x$, $3x$, $4x$, $5x$, and $6x$, respectively.

The sum of the probabilities of rolling each number must equal 1, so

$x+2x+3x+4x+5x+6x=1$

$21x=1$

$x=\frac{1}{21}$

So the probabilities of rolling a $1$, $2$, $3$, $4$, $5$, and $6$ are respectively $\frac{1}{21}, \frac{2}{21}, \frac{3}{21}, \frac{4}{21}, \frac{5}{21}$, and $\frac{6}{21}$. 

The possible combinations of two rolls that total $7$ are: $(1,6) ; (2,5) ; (3,4) ; (4,3) ; (5,2) ; (6,1)$

The probability P of rolling a total of $7$ on the two dice is equal to the sum of the probabilities of rolling each combination. 

$P = \frac{1}{21}\cdot\frac{6}{21}+\frac{2}{21}\cdot\frac{5}{21}+\frac{3}{21}\cdot\frac{4}{21}+\frac{4}{21}\cdot\frac{3}{21}+\frac{5}{21}\cdot\frac{2}{21}+\frac{6}{21}\cdot\frac{1}{21}=\frac{8}{63} \Rightarrow C$

%
\item%
(Alcumus solution)
On each die the probability of rolling $k$, for $1\leq k\leq 6$, is $\frac{k}{1+2+3+4+5+6}=\frac{k}{21}.$There are six ways of rolling a total of 7 on the two dice, represented by the ordered pairs $(1,6)$, $(2,5)$, $(3,4)$, $(4,3)$, $(5,2)$, and $(6,1)$. Thus the probability of rolling a total of 7 is $\frac{1\cdot6+2\cdot5+3\cdot4+4\cdot3+5\cdot2+6\cdot1}{21^2}=\frac{56}{21^2}=\boxed{\frac{8}{63}}.$

%
\item%
There are $6$ ways to get the sum of $7$ of the dice. Let's do case by case.

Case $1$: $\frac {1}{21} \cdot \frac {6}{21} = \frac {6}{441}$.

Case $2$: $\frac {2}{21} \cdot \frac {5}{21} = \frac {10}{441}$. 

Case $3$: $\frac {3}{21} \cdot \frac {4}{21} = \frac {12}{441}$. 

The rest of the cases are symmetric to these cases above. We have $2 \cdot \frac {28}{441}$. We have $\frac {56}{441} = \frac {8}{63}$. Therefore, our answer is $\boxed {\frac {8}{63}}$

~Arcticturn

%
\item%
\href{https://youtu.be/IRyWOZQMTV8?t=3057}{https://youtu.be/IRyWOZQMTV8?t=3057}

~ pi\_is\_3.14

%
\end{enumerate}

%
\section*{Problem 24}%
\label{sec:Problem24}%
\begin{enumerate}%
\item%

\begin{center}
\begin{asy}
	import olympiad; import cse5;   /* Made by MRENTHUSIASM */ size(200); draw(polygon(6)); pair A, B, C, D, E, F, X, Y, Z, M, N, O, P, Q, R; A = dir(120); B = dir(60); C = dir(0); D = dir(300); E = dir(240); F = dir(180); X = midpoint(A--B); Y = midpoint(C--D); Z = midpoint(E--F); M = intersectionpoint(A--E,X--Z); N = intersectionpoint(A--C,X--Y); O = intersectionpoint(C--E,Y--Z); P = intersectionpoint(A--C,X--Z); Q = intersectionpoint(C--E,X--Y); R = intersectionpoint(A--E,Y--Z); fill(M--P--N--Q--O--R--cycle,mediumgray); dot("$A$",A,1.5*dir(A),linewidth(4)); dot("$B$",B,1.5*dir(B),linewidth(4)); dot("$C$",C,1.5*dir(C),linewidth(4)); dot("$D$",D,1.5*dir(D),linewidth(4)); dot("$E$",E,1.5*dir(E),linewidth(4)); dot("$F$",F,1.5*dir(F),linewidth(4)); dot("$X$",X,1.5*dir(X),linewidth(4)); dot("$Y$",Y,1.5*dir(Y),linewidth(4)); dot("$Z$",Z,1.5*dir(Z),linewidth(4)); dot(M^^N^^O^^P^^Q^^R,linewidth(4)); draw(A--C--E--cycle^^X--Y--Z--cycle); 
\end{asy}
\end{center}
~MRENTHUSIASM

%
\item%

\begin{center}
\begin{asy}
	import olympiad; import cse5;   /* Made by MRENTHUSIASM */ size(200); draw(polygon(6)); pair A, B, C, D, E, F, X, Y, Z, M, N, O, P, Q, R; A = dir(120); B = dir(60); C = dir(0); D = dir(300); E = dir(240); F = dir(180); X = midpoint(A--B); Y = midpoint(C--D); Z = midpoint(E--F); M = intersectionpoint(A--E,X--Z); N = intersectionpoint(A--C,X--Y); O = intersectionpoint(C--E,Y--Z); P = intersectionpoint(A--C,X--Z); Q = intersectionpoint(C--E,X--Y); R = intersectionpoint(A--E,Y--Z); fill(M--P--N--Q--O--R--cycle,mediumgray); dot("$A$",A,1.5*dir(A),linewidth(4)); dot("$B$",B,1.5*dir(B),linewidth(4)); dot("$C$",C,1.5*dir(C),linewidth(4)); dot("$D$",D,1.5*dir(D),linewidth(4)); dot("$E$",E,1.5*dir(E),linewidth(4)); dot("$F$",F,1.5*dir(F),linewidth(4)); dot("$X$",X,1.5*dir(X),linewidth(4)); dot("$Y$",Y,1.5*dir(Y),linewidth(4)); dot("$Z$",Z,1.5*dir(Z),linewidth(4)); dot("$M$",M,1.5*dir(165),linewidth(4)); dot("$N$",N,1.5*dir(45),linewidth(4)); dot("$O$",O,1.5*dir(-75),linewidth(4)); dot("$P$",P,1.5*dir(105),linewidth(4)); dot("$Q$",Q,1.5*dir(-15),linewidth(4)); dot("$R$",R,1.5*dir(-135),linewidth(4)); draw(A--C--E--cycle^^X--Y--Z--cycle); draw(M--N--O--cycle,dashed); 
\end{asy}
\end{center}

The desired area (hexagon $MPNQOR$) consists of an equilateral triangle ($\triangle MNO$) and three right triangles ($\triangle MPN,\triangle NQO,$ and $\triangle ORM$).

Notice that $\overline {AD}$ (not shown) and $\overline {BC}$ are parallel. $\overline {XY}$ divides transversals $\overline {AB}$ and $\overline {CD}$ into a $1:1$ ratio (This can be shown by similar triangles.). Thus, it must also divide transversal $\overline {AC}$ and transversal $\overline {CO}$ into a $1:1$ ratio. By symmetry, the same applies for $\overline {CE}$ and $\overline {EA}$ as well as $\overline {EM}$ and $\overline {AN}.$

In $\triangle ACE,$ we see that $\frac{[MNO]}{[ACE]} = \frac{1}{4}$ and $\frac{[MPN]}{[ACE]} = \frac{1}{8}.$ Our desired area becomes \[\left(\frac{1}{4}+3 \cdot \frac{1}{8}\right) \cdot \frac{\sqrt{3}^2 \cdot \sqrt{3}}{4} = \frac {15}{32}\sqrt{3} = \boxed{\textbf{(C)}\ \frac {15}{32}\sqrt{3}}.\]

%
\item%
Instead of directly finding the desired hexagonal area, $\triangle XYZ$ can be found. It consists of three triangles and the desired hexagon. Given triangle rotational symmetry, the three triangles are congruent. See that $\triangle XYZ$ and $\triangle ACE$ are equilateral, so $m\angle PXN=60,$ so $m\angle AXP = \frac{180-60}{2}=60.$ As $\overline {AC}$ is a transversal running through $\overline {FC}$ (use your imagination) and $\overline {AB},$ we have $m\angle BAC=m\angle FCA = \frac{m\angle ACE}{2}=30.$

Then, $\triangle APX$ is a $30$-$60$-$90$ triangle. By HL congruence, $\triangle APX \cong \triangle NPX.$ Note that $AX=\frac{1}{2}.$ Then, the area of $\triangle PXN$ is $\frac{\sqrt{3}}{32}.$ There are three such triangles for a total area of $\triangle XYZ$ is $\frac{3\sqrt{3}}{32}.$ Find the side of $\triangle XYZ$ to be $\frac{3}{2},$ so the area is $\frac{9\sqrt{3}}{16}.$

The answer is \[\frac{9\sqrt{3}}{16}-\frac{3\sqrt{3}}{32}=\boxed{\textbf{(C)}\ \frac {15}{32}\sqrt{3}}.\]
~BJHHar

%
\item%

\begin{center}
\begin{asy}
	import olympiad; import cse5;   /* Made by MRENTHUSIASM */ size(200); draw(polygon(6)); pair A, B, C, D, E, F, X, Y, Z, M, N, O, P, Q, R; A = dir(120); B = dir(60); C = dir(0); D = dir(300); E = dir(240); F = dir(180); X = midpoint(A--B); Y = midpoint(C--D); Z = midpoint(E--F); M = intersectionpoint(A--E,X--Z); N = intersectionpoint(A--C,X--Y); O = intersectionpoint(C--E,Y--Z); P = intersectionpoint(A--C,X--Z); Q = intersectionpoint(C--E,X--Y); R = intersectionpoint(A--E,Y--Z); fill(M--P--N--Q--O--R--cycle,mediumgray); dot("$A$",A,1.5*dir(A),linewidth(4)); dot("$B$",B,1.5*dir(B),linewidth(4)); dot("$C$",C,1.5*dir(C),linewidth(4)); dot("$D$",D,1.5*dir(D),linewidth(4)); dot("$E$",E,1.5*dir(E),linewidth(4)); dot("$F$",F,1.5*dir(F),linewidth(4)); dot("$X$",X,1.5*dir(X),linewidth(4)); dot("$Y$",Y,1.5*dir(Y),linewidth(4)); dot("$Z$",Z,1.5*dir(Z),linewidth(4)); dot("$M$",M,1.5*dir(165),linewidth(4)); dot("$N$",N,1.5*dir(45),linewidth(4)); dot("$O$",O,1.5*dir(-75),linewidth(4)); dot("$P$",P,1.5*dir(105),linewidth(4)); dot("$Q$",Q,1.5*dir(-15),linewidth(4)); dot("$R$",R,1.5*dir(-135),linewidth(4)); draw(A--C--E--cycle^^X--Y--Z--cycle); draw(M--N--O--cycle,dashed); 
\end{asy}
\end{center}

Now, if we look at the figure, we can see that the complement of the hexagon we are trying to find is composed of $3$ isosceles trapezoids (namely $AXZF,CYXB,$ and $EZYD$) and $3$ right triangles (namely $\triangle XPN,\triangle YQO,$ and $\triangle ZRM$).

Finding the trapezoid's area, we know that one base of each trapezoid is just the side length of the hexagon, which is $1,$ and the other base is $\frac{3}{2}$ (it is halfway in between the side and the longest diagonal, which has length $2$) with a height of $\frac{\sqrt{3}}{4}$ (by using the Pythagorean Theorem and the fact that it is an isosceles trapezoid) to give each trapezoid having an area of $\frac{5\sqrt{3}}{16}$ for a total area of $\frac{15\sqrt{3}}{16}.$ (Alternatively, we could have calculated the area of hexagon $ABCDEF$ and subtracted the area of $\triangle XYZ,$ which, as we showed before, had a side length of $\frac{3}{2}$).

Now, we need to find the area of each of the small triangles, which, if we look at the triangle that has a vertex on $X,$ is similar to the triangle with a base of $YC = \frac12.$ Using similar triangles, we calculate the base to be $\frac{1}{4}$ and the height to be $\frac{\sqrt{3}}{4}$ giving us an area of  $\frac{\sqrt{3}}{32}$ per triangle, and a total area of $\frac{3\sqrt{3}}{32}.$ Adding the two areas together, we get $\frac{15\sqrt{3}}{16} + \frac{3\sqrt{3}}{32} = \frac{33\sqrt{3}}{32}.$ Finding the total area, we get $6 \cdot 1^2 \cdot \frac{\sqrt{3}}{4}=\frac{3\sqrt{3}}{2}.$ Taking the complement, we get $\frac{3\sqrt{3}}{2} - \frac{33\sqrt{3}}{32} = \boxed{\textbf{(C)}\ \frac {15}{32}\sqrt{3}}.$

%
\item%

\begin{center}
\begin{asy}
	import olympiad; import cse5;   /* Made by MRENTHUSIASM */ size(200); draw(polygon(6)); pair A, B, C, D, E, F, X, Y, Z, M, N, O, P, Q, R; A = dir(120); B = dir(60); C = dir(0); D = dir(300); E = dir(240); F = dir(180); X = midpoint(A--B); Y = midpoint(C--D); Z = midpoint(E--F); M = intersectionpoint(A--E,X--Z); N = intersectionpoint(A--C,X--Y); O = intersectionpoint(C--E,Y--Z); P = intersectionpoint(A--C,X--Z); Q = intersectionpoint(C--E,X--Y); R = intersectionpoint(A--E,Y--Z); fill(M--P--N--Q--O--R--cycle,mediumgray); dot("$A$",A,1.5*dir(A),linewidth(4)); dot("$B$",B,1.5*dir(B),linewidth(4)); dot("$C$",C,1.5*dir(C),linewidth(4)); dot("$D$",D,1.5*dir(D),linewidth(4)); dot("$E$",E,1.5*dir(E),linewidth(4)); dot("$F$",F,1.5*dir(F),linewidth(4)); dot("$X$",X,1.5*dir(X),linewidth(4)); dot("$Y$",Y,1.5*dir(Y),linewidth(4)); dot("$Z$",Z,1.5*dir(Z),linewidth(4)); dot("$M$",M,1.5*dir(165),linewidth(4)); dot("$N$",N,1.5*dir(45),linewidth(4)); dot("$O$",O,1.5*dir(-75),linewidth(4)); dot("$P$",P,1.5*dir(105),linewidth(4)); dot("$Q$",Q,1.5*dir(-15),linewidth(4)); dot("$R$",R,1.5*dir(-135),linewidth(4)); draw(A--C--E--cycle^^X--Y--Z--cycle); 
\end{asy}
\end{center}

We could also subtract $\triangle APM,\triangle CQN,$ and $\triangle ERO$ from $\triangle ACE.$

Since $\angle BAF = 120^{\circ}$ and $\angle BAC = \angle FAE = 30^{\circ},$ we have $\angle CAE = \angle BAF-\angle BAC-\angle FAE=60^{\circ}.$

Since $AX=BX$ and $FZ=EZ,$ we have $AF \parallel XZ,$ from which $\angle AMX= \angle FAM = 30^{\circ}.$

We can show that $\triangle APM$ is $30$-$60$-$90$ using a similar method, $\triangle CQN$ and $\triangle ERO$ are also $30$-$60$-$90.$

Since $AC=CE=AE=\sqrt{3},$ we have $[ACE]=AC^2 \cdot \frac{\sqrt{3}}{4}=3 \cdot \frac{\sqrt{3}}{4} = \frac{3 \sqrt{3}}{4}.$

Since $AX= \frac{1}{2}$ and $AP = AX \cdot \frac{\sqrt{3}}{2}= \frac{1}{2} \cdot \frac{\sqrt{3}}{2} = \frac{\sqrt{3}}{4},$ we have $PM = AP \cdot \sqrt{3} = \frac{\sqrt{3}}{4} \cdot \sqrt{3} = \frac{3}{4}.$

Note that \[[APM]=[CQN]=[ERO]=\frac{1}{2} \cdot AP \cdot PM = \frac{1}{2} \cdot \frac{\sqrt{3}}{4} \cdot \frac{3}{4} = \frac{3 \sqrt{3}}{32}.\]
Therefore, we get
\[[PNQORM]=[ACE]-[APM]-[CQN]-[ERO]=\frac{3 \sqrt{3}}{4} - 3 \cdot \frac{3 \sqrt{3}}{32} = \frac{24 \sqrt{3}}{32} - \frac{9 \sqrt{3}}{32} = \boxed{\textbf{(C)}\ \frac {15}{32}\sqrt{3}}.\]

~\href{https://artofproblemsolving.com/wiki/index.php/User:Isabelchen}{https://artofproblemsolving.com/wiki/index.php/User:Isabelchen}

%
\item%
Dividing $\triangle MNO$ into two right triangles congruent to $\triangle PMN,$ we see that $[MPNQOR]=\dfrac{5}{8}[ACE].$ Because $[ACE] = \dfrac{1}{2}[ABCDEF],$ we have $[MPNQOR]=\dfrac{5}{16}[ABCDEF].$ From here, you should be able to tell that the answer will have a factor of $5,$ and $\boxed{\textbf{(C)} \frac {15}{32}\sqrt{3}}$ is the only answer that has a factor of $5.$ However, if you want to actually calculate the area, you would calculate $[ABCDEF]$ to be $6 \cdot \dfrac{\sqrt{3}}{2 \cdot 2} = \dfrac{3\sqrt{3}}{2},$ so $[MPNQOR] = \dfrac{5}{16} \cdot \dfrac{3\sqrt{3}}{2} = \boxed{\textbf{(C)}\ \frac {15}{32}\sqrt{3}}.$

%
\item%
We partition hexagon $ABCDEF$ into $48$ congruent $30^\circ\text{-}60^\circ\text{-}90^\circ$ triangles, as shown below:

\begin{center}
\begin{asy}
	import olympiad; import cse5;   /* Made by MRENTHUSIASM */ size(200); pair A, B, C, D, E, F, X, Y, Z, M, N, O, P, Q, R; A = dir(120); B = dir(60); C = dir(0); D = dir(300); E = dir(240); F = dir(180); X = midpoint(A--B); Y = midpoint(C--D); Z = midpoint(E--F); M = intersectionpoint(A--E,X--Z); N = intersectionpoint(A--C,X--Y); O = intersectionpoint(C--E,Y--Z); P = intersectionpoint(A--C,X--Z); Q = intersectionpoint(C--E,X--Y); R = intersectionpoint(A--E,Y--Z); fill(M--P--N--Q--O--R--cycle,mediumgray); draw(A--D^^B--E^^C--F^^X--Y--Z--cycle^^midpoint(A--F)--midpoint(B--C)--midpoint(D--E)--cycle,red); draw(A--C--E--cycle^^M--N--O--cycle^^M--midpoint(F--Z)^^M--F+1/4*(A-F)^^N--midpoint(X--B)^^N--B+1/4*(C-B)^^O--midpoint(Y--D)^^O--D+1/4*(E-D),blue); draw(polygon(6)); dot("$A$",A,1.5*dir(A),linewidth(4)); dot("$B$",B,1.5*dir(B),linewidth(4)); dot("$C$",C,1.5*dir(C),linewidth(4)); dot("$D$",D,1.5*dir(D),linewidth(4)); dot("$E$",E,1.5*dir(E),linewidth(4)); dot("$F$",F,1.5*dir(F),linewidth(4)); dot("$X$",X,1.5*dir(X),linewidth(4)); dot("$Y$",Y,1.5*dir(Y),linewidth(4)); dot("$Z$",Z,1.5*dir(Z),linewidth(4)); dot(M^^N^^O^^P^^Q^^R,linewidth(4)); 
\end{asy}
\end{center}
Let the brackets denote areas. Note that the desired region contains $15$ of the $48$ small triangles, so the answer is \[\frac{15}{48}[ABCDEF]=\frac{15}{48}\cdot\frac{3\sqrt3}{2}=\boxed{\textbf{(C)}\ \frac {15}{32}\sqrt{3}}.\]
~AlexLikeMath ~MRENTHUSIASM

%
\item%
Notice, the area of the convex hexagon formed through the intersection of the $2$ triangles can be found by finding the area of the triangle formed by the midpoints of the sides and subtracting the smaller triangles that are formed by the region inside this triangle but outside the other triangle. First, let's find the area of the triangle formed by the midpoint of the sides. Notice, this is an equilateral triangle, thus all we need is to find the length of its side. 
To do this, we look at the isosceles trapezoid outside this triangle but inside the outer hexagon. Since the interior angle of a regular hexagon is $120^\circ$ and the trapezoid is isosceles, we know that the angle opposite is $60^\circ,$ and thus the side length of this triangle is $1+2\left(\frac{1}{2}\cos 60^\circ\right)=1+\frac{1}{2}=\frac{3}{2}.$ So the area of this triangle is $\frac{\sqrt{3}}{4}s^2=\frac{9\sqrt{3}}{16}.$

Now let's find the area of the smaller triangles. Notice, $\triangle ACE$ cuts off smaller isosceles triangles from the outer hexagon. The base of these isosceles triangles is perpendicular to the base of the isosceles trapezoid mentioned before, thus we can use trigonometric ratios to find the base and height of these smaller triangles, which are all congruent due to the rotational symmetry of a regular hexagon. The area is then $\frac{1}{2}\left(\frac{1}{2}\cos 60^\circ\right)\left(\frac{1}{2}\sin 60^\circ\right)=\frac{\sqrt{3}}{32}$ and the sum of the areas is $3\cdot \frac{\sqrt{3}}{32}=\frac{3\sqrt{3}}{32}.$

Therefore, the area of the convex hexagon is $\frac{9\sqrt{3}}{16}-\frac{3\sqrt{3}}{32}=\frac{18\sqrt{3}}{32}-\frac{3\sqrt{3}}{32}=\boxed{\textbf{(C)}\ \frac {15}{32}\sqrt{3}}.$

%
\item%
If we try to coordinate bash this problem, it's going to look very ugly with a lot of radicals. However, we can alter and skew the diagram in such a way that all ratios of lengths and areas stay the same while making it a lot easier to work with. Then, we can find the ratio of the area of the wanted region to the area of $ABCDEF$ then apply it to the old diagram.

\begin{center}
\begin{asy}
	import olympiad; import cse5;   unitsize(1cm); draw((0,0)--(4,0)--(6,3.464)--(2,3.464)--(0,0)); draw((2,0)--(1,1.732)); draw((5,1.732)--(4,3.464)); draw((1.5, 0.866)--(3, 3.464)--(4.5, 0.866)--cycle); draw((2,0)--(2,3.464)--(5,1.732)--cycle); 
\end{asy}
\end{center}


\begin{center}
\begin{asy}
	import olympiad; import cse5;   unitsize(1cm); fill((1,4)--(1,3.5)--(2,3)--cycle,red); fill((1,1)--(1.5,1)--(1,2)--cycle,red); fill((3,1)--(3.5,1.5)--(4,1)--cycle,red); draw((1,0)--(1,4),gray(.7)); draw((2,0)--(2,4),gray(.7)); draw((3,0)--(3,4),gray(.7)); draw((0,1)--(4,1),gray(.7)); draw((0,2)--(4,2),gray(.7)); draw((0,3)--(4,3),gray(.7)); draw((0,0)--(4,0)--(4,4)--(0,4)--(0,0)); draw((2,0)--(0,2)); draw((4,2)--(2,4)); draw((1,1)--(1,4)--(4,1)--cycle); draw((0,4)--(2,0)--(4,2)--cycle); 
\end{asy}
\end{center}

The isosceles right triangle with a leg length of $3$ in the new diagram is $\triangle XYZ$ in the old diagram. We see that if we want to take the area of the new hexagon, we must subtract $\frac{3}{4}$ from the area of $\triangle XYZ$ (the red triangles), giving us $\frac{15}{4}.$ However, we need to take the ratio of this area to the area of $ABCDEF,$ which is $\frac{\frac{15}{4}}{12}=\frac{5}{16}.$ Now we know that our answer is $\frac{5}{16} \cdot \frac{3\sqrt{3}}{2}=\boxed{\textbf{(C)}\ \frac {15}{32}\sqrt{3}}.$

%
\item%
\href{https://www.youtube.com/watch?v=yDbn9Mx2myw}{https://www.youtube.com/watch?v=yDbn9Mx2myw}

%
\end{enumerate}

%
\section*{Problem 25}%
\label{sec:Problem25}%
\begin{enumerate}%
\item%
As the red circles move about segment $AB$, they cover the area we are looking for.
On the left side, the circle must move around pivoted on $B$.
On the right side, the circle must move pivoted on $A$
However, at the top and bottom, the circle must lie on both A and B, giving us our upper and lower bounds.

This egg-like shape is $S$.

The area of the region can be found by dividing it into several sectors, namely

\begin{align*} A &= 2(\mathrm{Blue\ Sector}) + 2(\mathrm{Red\ Sector}) - 2(\mathrm{Equilateral\ Triangle}) \\ A &= 2\left(\frac{120^\circ}{360^\circ} \cdot \pi (2)^2\right) + 2\left(\frac{60^\circ}{360^\circ} \cdot \pi (1)^2\right) - 2\left(\frac{(1)^2\sqrt{3}}{4}\right) \\ A &= \frac{8\pi}{3} + \frac{\pi}{3} - \frac{\sqrt{3}}{2} \\ A &= 3\pi - \frac{\sqrt{3}}{2} \Longrightarrow \textbf {(C)}\end{align*}

%
\end{enumerate}

%
\end{document}